\documentclass[10pt]{article}
\usepackage[margin=1in]{geometry}
\usepackage{amsmath, amssymb, cancel, xcolor}

\newcommand\hcancel[2][black]{\setbox0=\hbox{$#2$}%
\rlap{\raisebox{.45\ht0}{\textcolor{#1}{\rule{\wd0}{1pt}}}}#2} 

\begin{document}

\begin{flushleft}
    Brandon Szeto \\
    Professor Kenneth Zeger \\
	ECE 109 \\
\end{flushleft}

\begin{center}
	\Large \textbf{Week 2, Lecture 01-17-23}
\end{center}
\normalsize

\begin{itemize}
    \item[\textbf{\underline{Example:}}] \textbf{Given a box with 6 pennies and 8 quarters, pick
        5 of the coins at random (without replacement). What is the probability
    that we choose 2 pennies and 3 quarters?} \\
    There are a total of $\binom{14}{5}$ 5-tuples of coins. How many of these
    choices are "good"? i.e. 2 pennies, 3 quarters. There are $\binom{6}{2}$
    ways of picking 2 pennies and $\binom{8}{3}$ ways of picking 3 quarters.
    $$\begin{aligned}
        &\therefore \text{the total number of good 5-tuples is the product} \binom{6}{2}
    \binom{8}{3} \\
        &\therefore \text{using equiprobability (i.e. all 5-tuples have the same
    probability)} \\
        &\therefore \text{Probability} = \frac{\binom{6}{2}\binom{8}{3}}{\binom{14}{5}}
\end{aligned}$$
    \item[\textbf{\underline{Example:}}] \textbf{Toss a coin 3 times. What is
        the probability we get exactly 2 heads?} \\
        The sample space of all possible outcomes can be defined as:
        $$ S = \{HHH, HHT, HTH, HTT, THH, THT, TTH, TTT\}$$
        $$ P(\{ HHT, HTH, THH \}) = \frac{3}{8}$$
        \textbf{What is the probability we get
        exactly 2 heads, given that the first two flips are not both heads?}
        $$ S = \{\hcancel[red]{HHH}, \hcancel[red]{HHT}, HTH, HTT, THH, THT, TTH, TTT\}$$
        Now, there is only two possible outcomes (i.e. HTH and THH) and only 6
        to choose from. \\
        Let us define the events:
        $$ \begin{aligned}
            E &= \text{"Exactly two heads occur"}\\
            F &= \{HHT, HHH\}^c\\
        \end{aligned} $$
        Intuitively, the probability is $\frac{2}{6} = \frac{1}{3}$.
        $$\boxed{ \text{We write } P(E | F) \text{ to mean } P(E) \text{ given }
        P(F) }$$
        $$\boxed{ \text{\textbf{Definition:} If } P(F) > 0 \text{, then define }
        P(E | F) = \frac{P(EF)}{P(F)}}. $$
        This is also called the conditional probability of E given F and is
        intuitive given a venn diagram.

    \item[\textbf{\underline{Example:}}] \textbf{Roll 2 dice. Find the
        probability both dice are even given their sum is greater than or equal
    to 10.} \\
    Let us define the events:
        $$ \begin{aligned}
            E &= \text{Both dice are even}\\
            F &= \text{Sum is } \geq 10 \\
              &= \{ (6,6), (6,5), (5,6), (6,4), (4,6), (5,5) \}
        \end{aligned} $$
        We want $P(E | F)$.
        $$ \begin{aligned}
            EF &= \{(6,6), (6,4), (4,6)\}\\
            P(EF) &=  \frac{|EF|}{|S|} = \frac{3}{36} = \frac{1}{12} \\
            P(F) &=  \frac{|F|}{|S|} = \frac{6}{36} = \frac{1}{6} \\
            P(E|F) &=  \frac{P(EF)}{P(F)} = \frac{\frac{3}{36}}{\frac{6}{36}} =
            \frac{3}{6} = \frac{1}{2} \\
        \end{aligned} $$
        \textbf{Now find the probability the sum = 7, given the sum $\neq$ 6.}
    Let us define the events:
        $$ \begin{aligned}
            E &= \text{Sum = 7}\\
            F &= \text{Sum} \neq 6 \\
        \end{aligned} $$
        We want $P(E | F)$.
        $$ \begin{aligned}
            E &= \{(1,6), (6,1), (2,5), (5,2), (4,3), (3,4)\}\\
            F &= \{(1,5), (5,1), (4,2), (2,4), (3,3)\}^c \\
            P(F^c) &= \frac{5}{36} \\
            P(F)   &= 1 - \frac{5}{36} = \frac{31}{36} \\
        \end{aligned} $$
        \textbf{Note:} $E \subseteq F$ implies $EF = E$. Therefore,
        $$ \begin{aligned}
            P(EF) &= P(E) = \frac{6}{36} \\
            P(EF) &= \frac{P(EF)}{P(F)} = \frac{\frac{6}{36}}{\frac{31}{36}} =
            \frac{6}{31} \\
        \end{aligned} $$
\end{itemize}

\begin{flushleft}
\textbf{Special Cases:}
\begin{enumerate}
    \item If $E,F$ are disjoint, then $EF = 0$, so $P(EF) = 0$. \\
        $\therefore P(E|F) = \frac{P(EF)}{P(F)} = 0$
    \item If $E \subseteq F$, then $EF = E$. So $P(F|E) = \frac{P(EF)}{P(E)} =
        \frac{P(E)}{P(E)} = 1$
\end{enumerate}

\textbf{Potential useful property:}
\end{flushleft}
$$ \boxed{P(EF) = P(E|F)P(F) = P(F|E)P(E)}$$

\begin{itemize}
    \item[\textbf{\underline{Example:}}] \textbf{A box contains 3 blue, 4 red,
            and 7
        green marbles. One marble is chosen at random and it is not red. What is
    the probability that it is blue?}
\end{itemize}
    Let us define the events:
        $$ \begin{aligned}
            E &= \text{Marble is blue}\\
            F &= \text{Marble is not red} \\
        \end{aligned} $$
        We want $P(E | F)$.
        We know that
        $$ \begin{aligned}
            P(F^c) &= P(\text{Marble is red})\\
                   &= \frac{4}{3 + 4 + 7} = \frac{4}{14} \\
            P(F) &= 1 - \frac{4}{14} = \frac{10}{14} \\
        \end{aligned} $$
        We claim that $E \subseteq F$, as the even that a blue marble is chosen
        implies that the chosen marble is not red, whereas if the chosen marble
        is not red, this does not imply that the marble is blue. Resultantly,
        $$ \begin{aligned} 
            EF &= E \\
            P(EF) &= P(E) = \frac{3}{14} \\
        \end{aligned}$$
        $$ \therefore P(E|F) = \frac{P(EF)}{P(F)} =
        \frac{\frac{3}{14}}{\frac{10}{14}} = \frac{3}{10}$$

        \newpage

\begin{center}
	\Large \textbf{Week 2, Lecture 01-19-23}
\end{center}
\normalsize

\begin{flushleft}
    \textbf{Recall Conditional Probability:}
    $$ P(E|F) = \frac{P(EF)}{P(F)} $$
\end{flushleft}

\begin{flushleft}
    \textbf{Recall Axioms of Probability:}
    \begin{enumerate}
        \item $ 0 \leq P(E) \leq 1 $
        \item $ P(S) = 1 $
        \item If $E_1, E_2, ... $ are pairwise disjoint events, then $ P(E_1 \cup
            E_2 \cup ... E_n ) = \sum_n P(E_n) $
    \end{enumerate}
    \textbf{Axioms of Conditional Probability:}
    \begin{enumerate}
        \item $ 0 \leq P(E|F) \leq 1 $
        \item $ P(S|F) = 1 $
        \item If $E_1, E_2, ... $ are pairwise disjoint events, then $ P((E_1 \cup
            E_2 \cup ... E_n)|F) = \sum_n P(E_n|F) $
    \end{enumerate}
    \textbf{Partition Rule of Probability:}
    $$ \begin{aligned} 
        B &= BA \cup BA^c       \\
        P(B) &= P(BA \cup BA^c) \\
             &= P(BA) + P(BA^c) \\
             &= \boxed{P(B|A)P(A) + P(B|A^c)P(A^c)}
    \end{aligned} $$
    Can be viewed as $A$ being case 1 and $A^c$ being case 2. This can be
    further generalized to $n$ cases. Suppose $A_1, A_2, ... , A_n$ are events
    that partition $S$. ($A_i$'s are a disjoint cover of $S$).
    $$ \begin{aligned} 
        \text{disjoint: }& i \neq j \rightarrow A_iA_j = 0 \\
        \text{cover: }& S = A_1 \cup A_2 \cup ... A_n \\
    \end{aligned} $$
    $$ \begin{aligned} 
        B &= BA_1 \cup BA_2 \cup ... \cup BA_n \\
        P(B) &= P(BA_1 \cup BA_2 \cup ... \cup BA_n) \\
             &= P(BA_1) + P(BA_2) + ... + P(BA_n) \\
             &= \boxed{P(B|A_1)P(A_1) + P(B|A_2)P(A_2) + ... + P(B|A_n)P(A_n) }\\
    \end{aligned} $$
    This can be extended to conditional probability. Can be thought of as
    inserting the cnoditional event into the above partition rule of
    probability:
    $$ \boxed{P(B|A_1E)P(A_1|E) + P(B|A_2E)P(A_2|E) + ... + P(B|A_nE)P(A_n|E) }$$

    \begin{itemize}
\item[\textbf{\underline{Example:}}] \textbf{Roll 1 fair 6-sided die. If the die
    comes up $\geq 3$, then we win. If not, then we flip a fair coin and we win
if heads and lose if tails. What is the probability of winning?} \\
    Let us define the events:
        $$ \begin{aligned}
            E &= \text{Die is } \geq 3\\
            F &= \text{We win}\\
        \end{aligned} $$
        We want $P(F)$.
        $$ \begin{aligned}
            P(F) &= P(F|E)P(E) + P(F|E^c)P(E^c) \\
            P(E) &= P(\text{Die is 3,4,5 or 6}) = \frac{4}{6} = \frac{2}{3} \\
            P(F|E) &= 1 \\
            P(F|E^c) &= \frac{1}{2}\\
            P(E^c) &= 1 - P(E) = 1 - \frac{2}{3} = \frac{1}{3} \\
            \therefore P(F) &= (1 * \frac{2}{3}) + (\frac{1}{2} * \frac{1}{3}) =
            \frac{5}{6}
        \end{aligned}$$
        \textbf{What is the probability the die is $\geq 3$, given we won?} \\
        In this case, we want $P(E|F)$. Recall,
        $$ \begin{aligned}
            P(EF) &= P(F)P(E|F) \\
                  &= P(E)P(F|E)
        \end{aligned}$$
        From before, we know
        $$ \begin{aligned}
            P(E) &= \frac{2}{3} \\
            P(F|E) &= 1 \\
            P(F) &= \frac{5}{6} \\
        \end{aligned}$$
        Plugging in,
        $$ \begin{aligned} 
            P(F)P(E|F) &= P(E)P(F|E) \\
            \frac{5}{6} * P(E|F) &= \frac{2}{3} * 1 \\
            \therefore P(E|F) &= \frac{\frac{2}{3}}{\frac{5}{6}} = \frac{4}{5}
        \end{aligned}$$

\item[\textbf{\underline{Example:}}] \textbf{A box contains 2 red and 3 green
    marbles. Pick 2 marbles at random (without replacement). If both are green,
we win. If both are red, we lose. Otherwise, we pick a third marble and we win
if red, lose if green. What is the probability of winning?} \\
Let us define the events
        $$ \begin{aligned}
            W &= \text{We win} \\
            G &= \text{Both marbles are green}\\
            R &= \text{Both marbles are red}\\
        \end{aligned} $$
        We want $P(W)$. The three cases for winning are:
        $$ P(W) = P(W|G)P(G) + P(W|R)P(R) + P(W|G^cR^c)P(G^cR^c)$$
        We know:
        $$ \begin{aligned}
            P(W|G) &= 1 \\
            P(W|R) &= 0 \\
            P(W|G^cR^c) &= \frac{1}{3} \\
            P(G^cR^c) &= P((G \cup R)^c) \\
                      &= 1 - P(G \cup R)
                      &= 1 - P(G) -P(R)
                      &= 1 - \frac{3}{5} * \frac{2}{4} - \frac{2}{5} *
                      \frac{1}{4} = \frac{3}{5} \\
        \end{aligned} $$
        Plugging in,
        $$ P(W) = (1 * \frac{3}{10}) + (0 * ?) + (\frac{1}{3}\frac{3}{5}) =
        \frac{1}{2} $$
\end{itemize}

   \textbf{Chain Rule of Probability:} \\
   Using the definition of conditional probability, we know:
   $$ P(A_1A_2) = P(A_1)P(A_2|A_1) $$
   For 3 events:
   $$ \begin{aligned}
       P(A_1A_2A_3) &= P(A_1(A_2A_3)) \\
                    &= P(A_1)P(A_2A_3|A_1)
   \end{aligned} $$
   \textbf{Note:}
   $$ \begin{aligned}
       P(A_3|A_1A_2) &= \frac{P(A_1A_2A_3)}{P(A_1A_2)} \\
                     &= \frac{P(A_2A_3|A_1) \hcancel[red]{P(A_1)}}{P(A_2|A_1)
                     \hcancel[red]{P(A_1)}}
   \end{aligned}$$
   Substitute in for $P(A_2|A_1) \rightarrow$
   $$ P(A_1A_2A_3) = P(A_1)P(A_2|A_1)P(A_3|A_1A_2) $$
   Can keep going:
$$ \boxed{P(A_1A_2 ... A_n) = P(A_1)P(A_2|A_1)P(A_3|A_1A_2) ... P(A_n|A_1A_2 ...
A_{n-1})}$$

   \begin{itemize}
\item[\textbf{\underline{Example:}}] \textbf{Given a standard deck of 52 cards,
    what is the probability of picking an ace, and then a red card (no
replacement) ?} \\
Let us define the events
        $$ \begin{aligned}
            A &= \text{First card is an ace} \\
            R &= \text{Second card is red}\\
            B &= \text{First card is red}\\
        \end{aligned} $$
        We want $P(AR)$. We know $P(AR) = P(A)P(R|A) = \frac{1}{13}P(R|A)$.
        Therefore all we need to find is $P(R|A)$. This can be rewritten as:
        $$ \begin{aligned}
            P(R|A) &= P(R|BA)P(B|A) + P(R|B^cA)P(B^c|A) \\
                   &= (\frac{25}{51} * \frac{1}{2}) + (\frac{26}{51} *
                   \frac{1}{2}) = \frac{1}{2} \\
            \therefore P(AR) &= \frac{1}{13} * \frac{1}{2} = \frac{1}{26}
        \end{aligned}$$

   \end{itemize}


   \textbf{Baye's Formula:} \\
   Suppose $A_1, A_2, ... , A_n$ is a partition of sample space $S$. Suppose we
   know $P(B|A_i)$ for all $i$. We want $P(A_i|B)$. This can be written as:
   $$ P(A_i|B) = \frac{P(B|A_i)P(A_i)}{P(B)} $$
   \textbf{Note:}
   $$ P(B) = \sum_{j = 1}^{n}P(B|A_j)P(A_j) $$
   $$ \boxed{
       P(A_i|B) = \frac{P(B|A_i)P(A_i)}{\sum_{j = 1}^{n}P(B|A_j)P(A_j)}
   }$$

   \newpage

\begin{center}
	\Large \textbf{Discussion 2}
\end{center}
\normalsize


\end{flushleft}

\end{document}
