\documentclass[10pt]{article}
\usepackage[margin=1in]{geometry}
\usepackage{amsmath, amssymb, cancel, xcolor}

\newcommand\hcancel[2][black]{\setbox0=\hbox{$#2$}%
\rlap{\raisebox{.45\ht0}{\textcolor{#1}{\rule{\wd0}{1pt}}}}#2} 

\begin{document}

\begin{flushleft}
    Brandon Szeto \\
    Professor Kenneth Zeger \\
	ECE 109 \\
\end{flushleft}

\begin{center}
	\Large \textbf{Week 2, Lecture 01-17-23}
\end{center}
\normalsize

\begin{itemize}
    \item[\textbf{\underline{Example:}}] \textbf{Given a box with 6 pennies and 8 quarters, pick
        5 of the coins at random (without replacement). What is the probability
    that we choose 2 pennies and 3 quarters?} \\
    There are a total of $\binom{14}{5}$ 5-tuples of coins. How many of these
    choices are "good"? i.e. 2 pennies, 3 quarters. There are $\binom{6}{2}$
    ways of picking 2 pennies and $\binom{8}{3}$ ways of picking 3 quarters.
    $$\begin{aligned}
        &\therefore \text{the total number of good 5-tuples is the product} \binom{6}{2}
    \binom{8}{3} \\
        &\therefore \text{using equiprobability (i.e. all 5-tuples have the same
    probability)} \\
        &\therefore \text{Probability} = \frac{\binom{6}{2}\binom{8}{3}}{\binom{14}{5}}
\end{aligned}$$
    \item[\textbf{\underline{Example:}}] \textbf{Toss a coin 3 times. What is
        the probability we get exactly 2 heads?} \\
        The sample space of all possible outcomes can be defined as:
        $$ S = \{HHH, HHT, HTH, HTT, THH, THT, TTH, TTT\}$$
        $$ P(\{ HHT, HTH, THH \}) = \frac{3}{8}$$
        \textbf{What is the probability we get
        exactly 2 heads, given that the first two flips are not both heads?}
        $$ S = \{\hcancel[red]{HHH}, \hcancel[red]{HHT}, HTH, HTT, THH, THT, TTH, TTT\}$$
        Now, there is only two possible outcomes (i.e. HTH and THH) and only 6
        to choose from. \\
        Let us define the events:
        $$ \begin{aligned}
            E &= \text{"Exactly two heads occur"}\\
            F &= \{HHT, HHH\}^c\\
        \end{aligned} $$
        Intuitively, the probability is $\frac{2}{6} = \frac{1}{3}$.
        $$\boxed{ \text{We write } P(E | F) \text{ to mean } P(E) \text{ given }
        P(F) }$$
        $$\boxed{ \text{\textbf{Definition:} If } P(F) > 0 \text{, then define }
        P(E | F) = \frac{P(EF)}{P(F)}}. $$
        This is also called the conditional probability of E given F and is
        intuitive given a venn diagram.

    \item[\textbf{\underline{Example:}}] \textbf{Roll 2 dice. Find the
        probability both dice are even given their sum is greater than or equal
    to 10.} \\
    Let us define the events:
        $$ \begin{aligned}
            E &= \text{Both dice are even}\\
            F &= \text{Sum is } \geq 10 \\
              &= \{ (6,6), (6,5), (5,6), (6,4), (4,6), (5,5) \}
        \end{aligned} $$
        We want $P(E | F)$.
        $$ \begin{aligned}
            EF &= \{(6,6), (6,4), (4,6)\}\\
            P(EF) &=  \frac{|EF|}{|S|} = \frac{3}{36} = \frac{1}{12} \\
            P(F) &=  \frac{|F|}{|S|} = \frac{6}{36} = \frac{1}{6} \\
            P(E|F) &=  \frac{P(EF)}{P(F)} = \frac{\frac{3}{36}}{\frac{6}{36}} =
            \frac{3}{6} = \frac{1}{2} \\
        \end{aligned} $$
        \textbf{Now find the probability the sum = 7, given the sum $\neq$ 6.}
    Let us define the events:
        $$ \begin{aligned}
            E &= \text{Sum = 7}\\
            F &= \text{Sum} \neq 6 \\
        \end{aligned} $$
        We want $P(E | F)$.
        $$ \begin{aligned}
            E &= \{(1,6), (6,1), (2,5), (5,2), (4,3), (3,4)\}\\
            F &= \{(1,5), (5,1), (4,2), (2,4), (3,3)\}^c \\
            P(F^c) &= \frac{5}{36} \\
            P(F)   &= 1 - \frac{5}{36} = \frac{31}{36} \\
        \end{aligned} $$
        \textbf{Note:} $E \subseteq F$ implies $EF = E$. Therefore,
        $$ \begin{aligned}
            P(EF) &= P(E) = \frac{6}{36} \\
            P(EF) &= \frac{P(EF)}{P(F)} = \frac{\frac{6}{36}}{\frac{31}{36}} =
            \frac{6}{31} \\
        \end{aligned} $$
\end{itemize}

\textbf{Special Cases:}
\begin{enumerate}
    \item If $E,F$ are disjoint, then $EF = 0$, so $P(EF) = 0$. \\
        $\therefore P(E|F) = \frac{P(EF)}{P(F)} = 0$
    \item If $E \subseteq F$, then $EF = E$. So $P(F|E) = \frac{P(EF)}{P(E)} =
        \frac{P(E)}{P(E)} = 1$
\end{enumerate}

\textbf{Potential useful property:}
$$ \boxed{P(EF) = P(E|F)P(F) = P(F|E)P(E)}$$

\begin{itemize}
    \item[\textbf{\underline{Example:}}] \textbf{A box contains 3 blue, 4 red,
            and 7
        green marbles. One marble is chosen at random and it is not red. What is
    the probability that it is blue?}
\end{itemize}
    Let us define the events:
        $$ \begin{aligned}
            E &= \text{Marble is blue}\\
            F &= \text{Marble is not red} \\
        \end{aligned} $$
        We want $P(E | F)$.
        We know that
        $$ \begin{aligned}
            P(F^c) &= P(\text{Marble is red})\\
                   &= \frac{4}{3 + 4 + 7} = \frac{4}{14} \\
            P(F) &= 1 - \frac{4}{14} = \frac{10}{14} \\
        \end{aligned} $$
        We claim that $E \subseteq F$, as the even that a blue marble is chosen
        implies that the chosen marble is not red, whereas if the chosen marble
        is not red, this does not imply that the marble is blue. Resultantly,
        $$ \begin{aligned} 
            EF &= E \\
            P(EF) &= P(E) = \frac{3}{14} \\
        \end{aligned}$$
        $$ \therefore P(E|F) = \frac{P(EF)}{P(F)} =
        \frac{\frac{3}{14}}{\frac{10}{14}} = \frac{3}{10}$$

        \newpage

\begin{center}
	\Large \textbf{Week 2, Lecture 01-19-23}
\end{center}
\normalsize

\end{document}
